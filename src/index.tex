\documentclass[12pt]{article}

\title{LaTeX Cheatsheet}
\author{author name}
%\date{01/01/2000}                        % if commented out, uses today's date

\usepackage{shared}

%\usepackage[top=2in, bottom=1.5in, left=1in, right=1in]{geometry}  % set margins
%\usepackage[margin=1in]{geometry}                                  % set all margins to the same value

\usepackage{boxedminipage}
\usepackage[a4paper]{geometry}            % A5 output. You should use this always for e reader users who have small screens (ereaders)
\usepackage{pdfpages}
\usepackage[notref,notcite]{showkeys}     % shows labels. Use labels with at most 7 chars or you fall off pdf in A5 kindle friendly mode
\usepackage{tabularx}         % Break line inside table cell: http://tex.stackexchange.com/questions/2441/how-to-add-a-forced-line-break-inside-a-table-cell
  % TODO get centering working
  %\usepackage{ragged2e}       % Center all tabularx: http://tex.stackexchange.com/questions/89166/centering-in-tabularx-and-x-columns
  %\renewcommand\tabularxcolumn[1]{>{\Centering}p{#1}}
\usepackage{wrapfig}

%TODO can this work??
%\newcommand{\inOut}[1]{\begin{verbatim}#1\end{verbatim}#1} %CANT use verbatim here, give up, people will read at source for now!!!
\newcommand{\inOut}[1]{#1}                %dummy inOut before I can find a nice way to show source and output in one go.

\begin{document}

\tableofcontents
\newpage

%\begin{comment}
\section{TODO}

  \begin{remark}
    Put TODOs as the first \label{TODO2} thing for development so you don't forget to do them.

    You may want to put those in order of urgency/difficulty here.

    Mark TODO location in \label{TODO1} the middle of text with labels TODO, then explain them here.

    Comment them out for release.
  \end{remark}

  \begin{itemize}
    \item TODO2 deal with this TODO at all costs
    \item TODO1 this TODO is not as important as the first one.
    \item TODO3
  \end{itemize}
  %\begin{comment}

\section{Title pages} \label{title-pages}

  \maketitle
  \newpage

  %#complex title page
  \begin{titlepage}
    COMPLEX TITLE PAGE

    PROMOTION 2009

    DURAN SANTILLI Ciro

    \vspace{40 mm}

    \begin{center}

      {\large RAPPORT DE STAGE DE RECHERCHE}\\[0.5cm]

      \underline{ \large \bfseries \itshape Flatness in control systems}

      \vspace{10 mm}

      \underline{ \bfseries NON CONFIDENTIEL }

      PUBLICATION

    \end{center}

    \vspace{40 mm}

    \underline{Option} : Mathématiques apliquées

    \underline{Champ de l'option} : Automatique et Contrôle

    \vspace{5 mm}

    INRIA Sophia Antipolis Méditerrané

    2004 Route des Lucioles - BP93

    06902 SOPHIA ANTIPOLIS Cedex France
  \end{titlepage}

  \newpage

\section{Abstract} \label{abstract}

  \begin{abstract}
    This attempts to cover every single feature of LaTeX to its minute details.

    It is intended to work with the \href{https://www.tug.org/texlive/acquire-iso.html}{TeX Live 2013 full ISO} without any additional packages installed or updated.
  \end{abstract}
  \newpage

\section{Symbols}\label{symbols}

  \begin{itemize}

    \item underscore:

      verb: \verb|a_b|

      Backslash: a\_b

    \item sharp / hash / number sign:

      verb: \verb|#|

      Backslash: \#

      TODO: what it the unescaped meaning?

    \item tilde:

      \url{http://tex.stackexchange.com/questions/9363/how-does-one-insert-a-backslash-or-a-tilde-into-latex}

      Center the tilde vertically: \url{http://tex.stackexchange.com/questions/312/correctly-typesetting-a-tilde}

      textasciitilde: Xx \textasciitilde

      verb: \verb|~|

      If unescaped, is a whitespace at which lines cannot break.

      Without tilde:

      mmmmmmmmmmmmmmmmmmmmmmmmmmmmmmmmmmmmmmmmmmmmmmmmmm nnnnnnnnnnnnnnnnnnnnnnnnnnnnnnnnnnnnnnnnnnnnnnnnnn 

      With tilde:

      mmmmmmmmmmmmmmmmmmmmmmmmmmmmmmmmmmmmmmmmmmmmmmmmmm~nnnnnnnnnnnnnnnnnnnnnnnnnnnnnnnnnnnnnnnnnnnnnnnnnn 

      Recommended before \lstinline|\ref| and object type: \lstinline|Table\ref{tab1}|.

    \item ellipsis:
    \begin{itemize}
      \item  \ldots   - horizontally at bottom of line
      \item  $\cdots$ - horizontally center of line (math mode only)
      \item  $\ddots$ - diagonal (math mode only)
      \item  $\vdots$ - vertical (math mode only)
    \end{itemize}

    \item international:

      With \lstinline|\usepackage[utf8]{inputenc}|:

      á à â ä

      Chinese does not work: %中文

    \item Derivatives:

      $$ \frac{dx(t)}{dt} $$
      $$ \frac{\partial x(u,v)}{\partial u} $$

    \item Vector arrow: $\vec{F}$

    \item
      Arrow with text above from amsmath:

        $A\xrightarrow{x^2}B$

      Arrow text above and below. \href{http://tex.stackexchange.com/questions/27545/custom-length-arrows-text-over-and-under}{Seems not to be on any package}:

        $A\xxrightarrow{x^1}[x^2]{x^3}B$

        $A\xxrightarrow{1000}{1}A$
  \end{itemize}

\section{Font}\label{font}

  \begin{itemize}
    \item {\Large Large brackets}
    \item
      \begin{Large}begin large environment\end{Large}
      {\Large large inside brackets}
      Possible sizes:
      \begin{lstlisting}
        \tiny
        \scriptsize
        \footnotesize
        \small
        \normalsize
        \large
        \Large
        \LARGE
        \huge
        \Huge
      \end{lstlisting}

      Arbitrary sizes are only possible for Type 1 fonts: \url{http://stackoverflow.com/questions/890127/how-to-set-latex-font-size-in-millimeter}
      \url{http://en.wikipedia.org/wiki/PostScript_fonts#Font_type}
    \item \textbf{textbf command}
    \item {\bf bf brackets bold}
    \item \textit{textit command}
    \item {\it it}
    \item {\it \bf it bf}
    \item {\bf \it bf it}
    \item \underline{underline underline}
    \item \uline{uline uline}. Requires \lstinline|\usepackage{ulem}|
  \end{itemize}

\section{Sections}\label{sections}

  \subsection{Subsection}\label{subsection}

    \subsubsection{Subsubsection}\label{subsubsection}

      \paragraph{Paragraph}\label{paragraph}

          Paragraph comes after subsubsection.

          To add a label and numbering to it, use:

          \begin{lstlisting}
            \setcounter{secnumdepth}{4}
          \end{lstlisting}

          To add it to the TOC use:

          \begin{lstlisting}
            \setcounter{tocdepth}{4}
          \end{lstlisting}

        \subparagraph{Subparagraph}\label{subparagraph}

          Subparagraph comes after paragraph.

          To add a label and numbering to it, use:

          \begin{lstlisting}
            \setcounter{secnumdepth}{5}
          \end{lstlisting}

          To add it to the toc, use:

          \begin{lstlisting}
            \setcounter{tocdepth}{5}
          \end{lstlisting}

\section{Positioning}\label{positioning}

  \begin{center}
    center environment

    center environment
  \end{center}

  {\centering
    centering group

    centering group: only works for single paragraphs
  }

  \begin{flushright}
    flushright environment

    flushright environment
  \end{flushright}

  TODO why does it not work?

  {\flushright flushright group}

  \subsection{Left and right at the same time}

    \begin{minipage}[t]{0.5\textwidth}
      \flushleft
      left1

      left2
    \end{minipage}
    \hfill
    \begin{minipage}[t]{0.5\textwidth}
      \flushright
      right1

      right2
    \end{minipage}

  \subsection{Wrap}\label{wrap}

    hfill float right:

    left \hfill float right \\
    left \hfill float right \\

  \begin{boxedminipage}{\textwidth}
    Multiple paragraphs

    with a border.
  \end{boxedminipage}

  \section{vspace}\label{wrap}

    vspace 1cm

    \vspace{1cm}

    vspace 1cm \\[1cm]

    vspace 1cm

  textwidth = width of the page where text can be (no margins)

\section{Formulas}\label{formulas}

  \begin{example}[Unumbered formula]\label{expFor2}
    \inOut{
      \begin{equation*}\label{eqFor2}
        \dot{x} = f(x,u)
      \end{equation*}
    }
  \end{example}\hrule

  \begin{example}[Numbered formula]\label{expFor1}
    \inOut{
      \begin{equation}\begin{aligned}\label{eqFor1}
        \dot{x} = f(x,u)
      \end{aligned}\end{equation}
    }
  \end{example}\hrule

  \subsection{Align at mulitple points}\label{alignMult}
    Can be done with alignat:

      \begin{alignat*}{3}
        & a + b && = c   && = c2 \\
        & d   && = e + f && = f2
      \end{alignat*}

  \begin{remark}\label{remFor1} Why I use equation + aligned by default
    There are simpler ways to write the equation such as backslash square brackets \textbackslash{}[\ldots\textbackslash{}],
    but if you use those you will soon notice that you will waste a long time modifying equations
    either to make them multiline, or to give them numbers, so it is just better to always use
    equation + aligned unless you have a reason not to do so.
  \end{remark}\hrule

  \subsection{Cases}\label{cases}
    Case function definition:
    \begin{equation}
      f(x) =
      \begin{cases}
        x & x \le 0 \\
        -x & x>0
      \end{cases}
    \end{equation}

\section{Floats}\label{floats}

  Floats are objects like tables or images that cannot be split across pages.
  Therefore, they may be displaced by LaTeX which can put text before them to
  not leave empty space.

  Since floats may have floated around, always refer to them via reference label pairs,
  like Float~\ref{float1}, \emph{never} as ``The next float''.

  \begin{table}
    \begin{tabular}{cc}
      1 \\ 1 \\ 1 \\ 1 \\ 1 \\
    \end{tabular}
    \caption{float1}
    \label{float1}
  \end{table}

  Don't worry too much about float positioning in early development: only think about
  it when you are almost done and the document won't be changing too much.

  Different options can increase the probability that floats won't float around too much.
  It may be a good idea to initially develop with \lstinline|[!htbp]| to see text close to floats
  and only remove it latter if needed. Table~\ref{floatHtpb} uses those options.

  \begin{table}
    \begin{tabular}{cc}
      !htbp \\ !htbp \\ !htbp \\ !htbp \\ !htbp \\
    \end{tabular}
    \caption{floatHtpb}
    \label{floatHtpb}
  \end{table}

  The float package offers the option \lstinline|H| which fixes the float at a position without reflow no matter what.

  TODO, what does each character mean exactly?

  \begin{itemize}
    \item \lstinline|!|
    \item \lstinline|h|
    \item \lstinline|t|
    \item \lstinline|b|
    \item \lstinline|p|
  \end{itemize}

  SE thread: \url{http://tex.stackexchange.com/questions/2275/keeping-tables-figures-close-to-where-they-are-mentioned}

\section{Figures}\label{figures}

  Figures, images, graphics.

  See Figure~\ref{fig-label}.

  \begin{figure}[htb]
    \includegraphics[width=5cm]{image.png}
    \caption{caption here}
    \label{fig-label}
  \end{figure}

  Allowed formats:

  \begin{itemize}
    \item PNG
    \item JPG
  \end{itemize}

  Non-allowed formats:

  \begin{itemize}
    \item GIF
  \end{itemize}

  You can append to the figure search path with the package \lstinline|graphicx| and the command \lstinline|\graphicspath|:

  \begin{lstlisting}
    \usepackage{graphicx}

    \graphicspath{{./media/}{/usr/share/latex/img/}}
  \end{lstlisting}

  Relative paths are taken relative to current directory.

  You can only use the \lstinline|\graphicspath| command once! Newer calls will erase the older ones.

%  Test media-gen:
%
%    \begin{figure}[htb]
%      \includegraphics[width=5cm]{media-gen.png}
%      \caption{caption here}
%      \label{fig-label}
%    \end{figure}

  \subsection{wrapfigure}\label{wrapfigure}%#float images

    wrapfigure from wrapfig package allows images to float right around text.
    Images float only around text that comes after them.

    a \\ a \\ a \\ a\\ a \\ a \\ a \\ a \\ a\\ a \\

    \begin{wrapfigure}{r}{0.5\textwidth}
      \includegraphics[height=2cm]{image.png}
      \caption{Floating chick B}
    \end{wrapfigure}

    b \\ b \\ b \\ b\\ b \\ b \\ b \\ b \\ b\\ b \\
    b \\ b \\ b \\ b\\ b \\ b \\ b \\ b \\ b\\ b \\

    \begin{wrapfigure}{r}{0.5\textwidth}
      \includegraphics[height=2cm]{image.png}
      \caption{Floating chick C}
    \end{wrapfigure}

    c \\ c \\ c \\ c\\ c \\ c \\ c \\ c \\ c\\ c \\
    c \\ c \\ c \\ c\\ c \\ c \\ c \\ c \\ c\\ c \\

    \begin{wrapfigure}{l}{0.5\textwidth}
      \includegraphics[height=2cm]{image.png}
      \caption{Floating chick D}
    \end{wrapfigure}

    d \\ d \\ d \\ d\\ d \\ d \\ d \\ d \\ d\\ d \\
    d \\ d \\ d \\ d\\ d \\ d \\ d \\ d \\ d\\ d \\

    \begin{wrapfigure}{r}{5cm}
      \begin{flushleft}
        Floating

        Paragraphs

        E
      \end{flushleft}
    \end{wrapfigure}

    TODO: how to remove top margin?

    e \\ e \\ e \\ e\\ e \\ e \\ e \\ e \\ e\\ e \\
    e \\ e \\ e \\ e\\ e \\ e \\ e \\ e \\ e\\ e \\

    Does not work one to left one to right:

    \begin{wrapfigure}{l}{0.3\textwidth}
        Floating

        Paragraphs

        l
    \end{wrapfigure}

    \begin{wrapfigure}{r}{0.5\textwidth}
        Floating

        Paragraphs

        r
    \end{wrapfigure}

\clearpage

\section{Tables}\label{table}

  Table~\ref{table1} is a simple table.

  \begin{table}[!htpb]
    \begin{tabular}{ccc}
      h1 & h2 & h3 \\
      \hline
      1 & 2 & 3 \\
      4 & 5 & 6 \\
      7 & 8 & 9 \\
    \end{tabular}
    \caption{table1}
    \label{table1}
  \end{table}

  For complex tables with label LABEL, create a LABEL.ods spreadsheet with same name as the label and use it to make the table, then copy paste to the tex.

  As with any other float (object that can change its position on the page to fit to content), always reference table labels when talking about tables, and never use expressions such as "the table" or "next table".

  Table~\ref{table-format} shows how to format individual table columns with \lstinline|\usepackage{array}|.

  \begin{table}[!htpb]
    \begin{tabular}{l c rp{1cm} p{1cm} >{\bf}c >{\it}c | c || c @{abc} c c}
      1 & 2 & 3 & 4 & 5 5 5 5 5 & 6 & 7 & 8 & 9 & 10 & 111111111111111111111111111111111111111111111 \\
      1 & 2 & 3 & 4 & 5 5 5 5 5 & 6 & 7 & 8 & 9 & 10 & 111111111111111111111111111111111111111111111 \\
    \end{tabular}
    \caption{table-format}
    \label{table-format}
  \end{table}

  \subsection{Break line inside table cell}\label{break-line-inside-table-cell}

    SE question: \url{http://tex.stackexchange.com/questions/2441/how-to-add-a-forced-line-break-inside-a-table-cell}

    The best solution we have found so far it is to use \lstinline|tabularx| and \lstinline|\newline|.
    as in Table~\ref{table-break-tabularx}. \emph{Warning}: for tabularx to work, you \emph{must}
    have an \lstinline|X| column!

    \begin{table}[!htpb]
      \begin{tabularx}{\textwidth}{lX}
          a & a \newline
              b \\
          a & a \newline
              b \\
      \end{tabularx}
      \caption{table-break-tabularx}
      \label{table-break-tabularx}
    \end{table}

    TODO how to get it automatically centered? floatrow does not work on it: \url{http://tex.stackexchange.com/questions/89166/centering-in-tabularx-and-x-columns}

    Table~\ref{table-break} shows how to format individual table columns without any extra packages:

    \begin{table}[!htpb]
      \begin{tabular}{l l}
        a & \begin{tabular}[t]{@{}l@{}} a \\ b \end{tabular} \\
        \hline
        a & \begin{tabular}[c]{@{}l@{}} a \\ b \end{tabular} \\
        \hline
        a & \begin{tabular}[b]{@{}l@{}} a \\ b \end{tabular} \\
        \hline
      \end{tabular}
      \caption{table-break}
      \label{table-break}
    \end{table}

  \subsection{Fixed width cell}\label{fixed-width-cell}

    \begin{tabular}{>{}m{10cm} | l}
      10cm default align & col2
    \end{tabular}

    \begin{tabular}{>{\centering}m{10cm} | l}
      10cm center & col2
    \end{tabular}

    \begin{tabular}{>{}m{10cm} | l}
      10cm center & col2
    \end{tabular}

\section{Include external PDF}\label{include-external-pdf}

  The following page shall be taken from an external PDF:

  \includepdf[pagecommand=\thispagestyle{plain}]{image.pdf}

  TODO: how to show the page number and make a label / hyperref to it? <http://tex.stackexchange.com/questions/25105/unable-to-link-to-inserted-pages-with-pdfpages>

\section{Comments}\label{comments}

    Next line will be commented out, and therefore invisible to output.
    \begin{comment}
    This line was commented off.
    \end{comment}

\section{Verbatim}\label{verbatim}

  Regular text.

  Block:

  \begin{verbatim}
  a
    b
c
  \end{verbatim}

  Inline \verb|~| tilde.

  There is no way to indent verbatim relative to its containing environment, but it can be done with \lstinline|lstlisting|.
  \url{http://tex.stackexchange.com/questions/164993/indent-content-of-verbatim-lstlisting-environment-relative-to-containing-envir#164999}

\section{Computer code}\label{computer-code}

  Use:

  \begin{lstlisting}
    \usepackage{lstlisting}
  \end{lstlisting}

  This is how you use it:

  \begin{lstlisting}
    if i in is:
        echo i
    else:
        echo -i
  \end{lstlisting}

  Inline listings: \lstinline|func_f(void a)| is a function.

  You can set properties for the entire document by using \lstinline|lstset| in the header.
  Sample usage:

  \begin{lstlisting}
    \lstset{
      %language=C,               % Code langugage
      basicstyle=\ttfamily,          % Code font, Examples.
      %commentstyle=\color{gray},       % Comments font
      %numbers=left,              % Line nums position
      %numberstyle=\tiny,           % Line-numbers fonts
      %stepnumber=1,              % Step between two line-numbers
      %numbersep=5pt,             % How far are line-numbers from code
      %backgroundcolor=\color{lightlightgray}, % Choose background color
      %frame=none,               % A frame around the code
      %tabsize=4,               % Default tab size
      %captionpos=b,              % Caption-position = bottom
      %breaklines=true,            % Automatic line breaking?
      %breakatwhitespace=false,        % Automatic breaks only at whitespace?
      %showspaces=false,            % Dont make spaces visible
      %showtabs=false,             % Dont make tabls visible
      %columns=flexible,            % Column format
      %keywordstyle=\color{OliveGreen},    % Keywords font ('*' = uppercase)
      %morekeywords={__global__, __device__}, % CUDA specific keywords
    }
  \end{lstlisting}

  Bad defaults which you should correct are:

  \begin{itemize}
    \item use the package \lstinline|upquote| so that quotation marks will appear
      correctly as ASCII characters: \lstinline|"'`|
    \item use the option \lstinline|basicstyle=\ttfamily| allows outputs monospace fonts:
      \begin{lstlisting}
        mmmmmmmmm
        iiiiiiiii
      \end{lstlisting}
    \item Use the \lstinline|literate={~} {$\sim$}{1}| option so that ASCII tildes can be used.
      \url{http://tex.stackexchange.com/questions/17266/how-to-insert-a-nice-tilde-in-a-lstlisting}
    \item TODO how to copy paste correct code from PDF? paste is always full of spaces.
  \end{itemize}

\section{References}\label{references}

  \lstinline|\label| refers to the smallest surrounding thing that is numbered,
  typically a section or a theorem environment.
  Therefore, if you simply put a label in a point paragraph,
  you don't normally get a link to that point of paragraph like you would in an HTML anchor.

  For HTML ID compatibility, use hyphen separation: def-a-definition-was-here.

  \begin{example} \label{expRef1}
    The next ref aims at the current example: \ref{expRef1}.
  \end{example}

  It is a good idea not to leave spaces between \lstinline|labels|.

  The floatrow package is a must: \lstinline|\usepackage{floatrow}| automatically centers all floats.

\section{Hyperlinks and URLs}\label{hyperlinks-urls}

  External links:

  \url{http://www.wikibooks.org}

  \href{http://www.wikibooks.org}{wikibooks home}

  Relative links:

  \url{./index.pdf}

  \href{./index.pdf}{link to index.pdf}

  Inner links in document:

  \hypertarget{label}{target caption}

  \hyperlink{label}{link caption}

  \section{variables and conditional compilation}

    \begin{lstlisting}
      \def\mycmd{1}

      \ifx\mycmd\undefined
        undefed
      \else
        \if\mycmd1
          defed, 1
        \else
          defed
        \fi
      \fi

      \def\mycmd{0}

      \ifx\mycmd\undefined
        undefed
      \else
        \if\mycmd1
          defed, 1
        \else
          defed
        \fi
      \fi
    \end{lstlisting}

\section{Quotations}\label{quotations}

  From \cite{Aa00}:

  \begin{quote}
    Some quoted text, single paragraph. Some quoted text, single paragraph. Some quoted text, single paragraph.
    Some quoted text, single paragraph. Some quoted text, single paragraph. Some quoted text, single paragraph.
    Some quoted text, single paragraph. Some quoted text, single paragraph. Some quoted text, single paragraph.
    Some quoted text, single paragraph. Some quoted text, single paragraph. Some quoted text, single paragraph.
    Some quoted text, single paragraph. Some quoted text, single paragraph. Some quoted text, single paragraph.
    Some quoted text, single paragraph. Some quoted text, single paragraph. Some quoted text, single paragraph.
    Some quoted text, single paragraph. Some quoted text, single paragraph. Some quoted text, single paragraph.
  \end{quote}

  Quotations: for use with longer quotations, of more than one paragraph,
  because it indents the first line of each paragraph.

  Verse: is for quotations where line breaks are important, such as poetry.
  Once in, new stanzas are created with a blank line, and new lines within
  a stanza are indicated using the newline command.

  If a line takes up more than one line on the page, then all subsequent lines
  are indented until explicitly separated with \lstinline|\\|.

\section{Citations and bibliography}\label{secCit}

  The best way is to use bibtex

  \begin{itemize}
    \item Aa00: \cite{Aa00}
    \item Aa01: \cite{Aa01}
  \end{itemize}

  You have to cite a reference before it appears in the bibliography

  In the bibliography command, use the same name as your .bib file.

\section{Bibliography}\label{secBib}

  \newpage
  \bibliography{index}

\end{document}
